\section{Conclusion}
In this paper, we present a novel \acl{oecp} that has the ability to measure the permittivity of a heterogeneous sample over a large frequency band (\qtyrange{0.5}{18}{\giga\hertz}), which is of particular interest for the field of microwave remote sensing.
In this regard, relevant satellite-based remote sensing sensor frequencies that are commonly used for snow monitoring were chosen in order to present the results.
The calibration of the probe shows that it is reliable over all the \qtyrange{0.5}{18}{\giga\hertz} range with minimal error (RMSE \(\approx1\)).
Tests on dry and wet stacked paper are used to determine that the penetration depth of the probe signal is between \qtyrange{0.5}{0.75}{\cm} depending on the frequency for the dry paper and around \qty{0.3}{\cm} for wet paper.
The probe was also used to measure the permittivity spectrum of dry sand (commercial) and the values were compared to other studies found in the literature.
Wet sand (commercial) and wet organic mesic soil sample (Cambridge Bay, Nunavut, Canada) samples were subjected to a temperature ramp varying from \qtyrange{-20}{20}{\degreeCelsius}.
Both samples show comparable permittivity/temperature curves where the frozen permittivity is close to \(\varepsilon^\prime = 5\) and thawed permittivity revolves around \(\varepsilon^\prime = 20\).

Further work using the probe is underway.
The \ac{oecp} permittivity measurements will be used to develop a large band soil permittivity model using the soil composition, humidity and temperature as input.
Such a model can be used alongside radiative transfer models such as \acl{smrt} to compute the emission or backscattering contribution of different surfaces in \acl{swe} retrieval algorithm.
This permittivity model will also be tested on real world data coming from the CryoSAR project.
A better understanding of the soil contribution to the total backscattering signal at Ku-band frequencies will enable to better understand the relationship between snow microstructure, snow height and snow density in the \ac{swe} inversion models that will be used in the \acl{tsmm}.

\section{Acknowledgement}
The authors acknowledge the support of the \acl{csa} (19FAWATA23) and the \acl{eccc} team for their help.
We also thank the Natural Sciences and Engineering Research Council of Canada (NSERC), Mitacs Globalink and the Bureau des Relations Internationales (BRI) of the Université du Québec à Trois-Rivières for funding this project. 
Special thanks go to Jean-Yves Delatage (U-Bordeaux) for his support during the soil freeze/thaw cycle analysis.
