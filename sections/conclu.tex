\section{Conclusion}
In this paper, we presented a novel \acl{oecp} that can measure the permittivity of a heterogeneous sample over a large frequency band (\qtyrange{0.5}{18}{\giga\hertz}), which is especially relevant for microwave remote sensing.
Here, we chose to only present the results for relevant satellite-based remote sensing sensor frequencies that are commonly used for snow monitoring.
The probe calibration shows its reliability over all the \qtyrange{0.5}{18}{\giga\hertz} range with minimal errors (RMSE \(\approx1\)).
Tests on dry and wet stacked paper were used to determine that the penetration depth of the probe signal is between \qtyrange{0.5}{0.75}{\cm} depending on the frequency for the dry paper and around \qty{0.3}{\cm} for wet paper.
The probe was also used to measure the permittivity spectrum of dry sand (commercial) and the values were compared to other studies found in the literature.
Wet commercial sand and wet organic mesic soil (Cambridge Bay, Nunavut, Canada) samples were subjected to a temperature ramp varying from \qtyrange{-20}{20}{\degreeCelsius}.
Both samples showed comparable permittivity/temperature curves where the frozen permittivity was around \(\varepsilon^\prime = 5\) and the thawed permittivity was around \(\varepsilon^\prime = 20\).

Further work using the probe is underway.
The \ac{oecp} permittivity measurements will be used to develop a large band soil permittivity model using the soil composition, humidity and temperature as input.
Such a model can be used alongside radiative transfer models like \acl{smrt} to compute the emission or backscattering contributions of different surfaces in \acl{swe} retrieval algorithms.
This permittivity model will also be tested on real-world data coming from the CryoSAR project.
Defining better soil contribution to the total backscattering signal at Ku-band frequencies will improve the understanding of the relationship between snow microstructure, snow height and snow density in the \ac{swe} inversion models that will be used in the \acl{tsmm}.

\section{Acknowledgement}
The authors acknowledge the support of the \acl{csa} (19FAWATA23) and the \acl{eccc} team for their help.
We also thank the Natural Sciences and Engineering Research Council of Canada (NSERC), Mitacs Globalink and the Bureau des relations internationales (BRI) of the Université du Québec à Trois-Rivières for funding this project. 
A special thank goes to Jean-Yves Delatage (U-Bordeaux) for his support during the soil freeze/thaw cycle analysis.
