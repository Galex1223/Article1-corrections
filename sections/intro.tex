\section{Introduction}

\IEEEPARstart{R}{ecent developments} in active microwave remote sensing showed the potential of \ac{sar} for monitoring important snow characteristics.
More precisely, studies demonstrated the high sensitivity of \ac{sar} backscattering in the Ku-band domain (\qtyrange{12}{18}{\giga\hertz}) to \acf{swe} \parencite{Shi2016,King2018,Rutter2019}.
Following \ac{swe} temporal and spatial variation in location where the snow coverage spans more than 6 months per year is of utmost importance for hydrological cycles \parencite{Zhang2005,Ayguen2020}, surface energy balance \parencite{Cohen2001,Box2019} and biogeochemical cycles \parencite{Zhang2005,Grogan2006}, among others.
The significance of seasonal snowmelt and \ac{swe} in delivering freshwater is of great importance for the well-being of Canadians, supporting a diverse array of economic sectors and sustaining ecosystems.
Simultaneously, it introduces risks through contributions to floods and the perpetuation of drought events.
The impact of these snow variables, combined with the recent \ac{sar} developments led to the beginning of a joint mission from \ac{csa} and \ac{eccc}, \acf{tsmm} \parencite{Derksen2019,Garnaud2019}, which aims to launch a dual Ku-band \ac{sar} (\qty{13.5}{\giga\hertz} and \qty{17.2}{\giga\hertz}) aboard a satellite for snow monitoring in the Northern Hemisphere.
Moreover, satellite Ku-band \ac{sar} observations would offer regular high resolution (few meters) \ac{swe} data for remote locations around the globe, under various meteorological conditions \parencite{Tsai2019}.
Indeed, Ku-band radar signals have low interactions with atmospheric particles such as water vapor in clouds and is completely independent of solar radiations.

However, there are still lots of challenges for \ac{swe} inversion using the Ku-band \ac{sar} backscatter signal.
Indeed, the backscattering signal is not only influenced by \ac{swe}, but also by other snowpack characteristics such as microstructure and liquid water content, and by the substrate under the snow (sea ice, fresh water ice, soil, etc.) and vegetation cover.
To overcome these obstacles, modelization of the signal via microwave radiative transfer models such as \ac{smrt} \parencite{Picard2018} is used to disentangle the effect from each contribution, enabling \ac{swe} retrieval from the backscattering signal.
Most of the existing models for active remote sensing backscatter were built for lower microwave frequencies (\(<\)\qty{6}{\giga\hertz}, \parencite{Longepe2009}), whereas \ac{smrt} was designed to work in a frequency range that encompasses the \ac{tsmm} mission dual Ku-band frequencies.
Moreover, \ac{smrt} was specifically conceived to include parameters such as snow microstructure and the substrate characteristics.
Nonetheless, \ac{smrt} developers note that there is a need for more precise wave diffusion models at snow/substrate interface \parencite{Picard2018}.
Soil contribution remains misrepresented by the model and brings some uncertainty in the snow signal produced \parencite{King2018}.
In order to model the soil contribution to the total backscattering signal, an important parameter to understand is the soil relative permittivity (hereafter referred to as permittivity).

Knowing the soil permittivity allows to model the soil influence over the total signal.
In current radiative transfer models targeting snow characteristics, frozen soil permittivity is often set to a constant close to that of dry soil \parencite{Hallikainen1985,Kerr2012}.
This is due to the lack of knowledge regarding the spatial and temporal variability of frozen soil permittivity.
Soil permittivity greatly depends on soil water content \parencite{Topp1980} where water permittivity is relatively high (\(\varepsilon^\prime \approx 80\) at L-band) compared to dry soil (\(\varepsilon^\prime \approx 2.5\) \parencite{Matzler1998}) or ice (\(\varepsilon^\prime \approx 3.2\) \parencite{Matzler1987}).
This means that having unfrozen water in the soil would greatly increase the permittivity, hence modifying the soil backscattering signal.
Previous work on soil dielectric properties have been made where the dielectric constant is measured as a function of either the temperature and/or the water content and the soil texture \parencite{Cihlar1974,Hoekstra1974,Bobrov2015,Kabir2020}.
Models for soil permittivity (called dielectric mixing models, \parencite{Dobson1985,Hallikainen1985,Mironov2009,Zhang2010}) have also previously been developed.%

In this context, the goal of this article is to present a novel \ac{oecp} with an operating frequency range from \SIrange{0.5}{18}{\giga\hertz}, able to measure the permittivity of a flat material or liquid.
The probe aperture (the measurement plane) is sufficiently large to integrate a representative average permittivity over non-homogeneous soil.
The experimental setup required to operate the probe is simple and easily portable, making it perfect for laboratory or field usage.
In this study, we present the novel \ac{oecp} and its precision over its full operating range as well as tests to estimate the probe signal penetration depth.
Then, liquids of known permittivity are used to compare the probe's results to theoretical values in order to validate the probe calibration.
Finally, a protocol was developed to measure the soil permittivity in a controlled varying temperature environment (from \qtyrange{-20}{20}{\degreeCelsius}).
Two different soil types (sand and arctic organic soil) were analyzed following the new protocol.
