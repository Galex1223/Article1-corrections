\section{Introduction}

\IEEEPARstart{R}{ecent developments} in active microwave remote sensing showed the potential of \ac{sar} for monitoring important snow characteristics.
More precisely, studies demonstrated that the \ac{sar} backscattering in the Ku-band domain (\qtyrange{12}{18}{\giga\hertz}) is highly sensible to \acf{swe} \parencite{Shi2016,King2018,Rutter2019}.
Following \ac{swe} temporal and spatial variations where the snow coverage spans more than 6 months per year is crucial to understand hydrological cycles \parencite{Zhang2005,Ayguen2020}, surface energy balance \parencite{Cohen2001,Box2019} and biogeochemical cycles \parencite{Zhang2005,Grogan2006}, among others.
Seasonal snowmelt and \ac{swe} are key in delivering freshwater for Canadians' well-being, for supporting diverse economic sectors and for sustaining ecosystems.
Simultaneously, snowmelt and \ac{swe} can impact floods and drought events.
The impact of these snow variables, combined with the recent \ac{sar} developments, led to the launch of a joint mission of the \ac{csa} and \ac{eccc}: the \acf{tsmm} \parencite{Derksen2019,Garnaud2019}, which aims to launch a dual Ku-band \ac{sar} (\qty{13.5}{\giga\hertz} and \qty{17.2}{\giga\hertz}) aboard a satellite to monitoring snow in the Northern Hemisphere.
Satellite Ku-band \ac{sar} observations would offer regular high resolution (few meters) \ac{swe} data for remote locations around the globe under various meteorological conditions \parencite{Tsai2019}.
Ku-band radar signals have low interactions with atmospheric particles such as water vapour in clouds and are completely independent of solar radiations.

However, a lot of challenges still exist for \ac{swe} inversion using the Ku-band \ac{sar} backscatter signal.
The backscattering signal is not only influenced by \ac{swe}, but also by other snowpack characteristics such as microstructure and liquid water content, and by the substrate under the snow (sea ice, fresh water ice, soil, etc.) and vegetation cover.
Modelling a signal in microwave radiative transfer models such as \ac{smrt} \parencite{Picard2018} allows to distinguish each contribution's effect, enabling \ac{swe} retrieval from the backscattering signal.
Most of the models for active remote sensing backscatter operate at lower microwave frequencies (\(<\)\qty{6}{\giga\hertz}, \parencite{Longepe2009}), whereas \ac{smrt} works in a frequency range that encompasses the \ac{tsmm} mission dual Ku-band frequencies.
\ac{smrt} was specifically designed to include parameters such as snow microstructure and substrate characteristics, but \ac{smrt} developers still note the need for more precise wave diffusion models at the snow/substrate interface \parencite{Picard2018}.
The model still misrepresents the soil contribution, which brings some uncertainty in the snow signal produced \parencite{King2018}.
To model the soil contribution to the total backscattering signal, an important parameter to understand is the soil relative permittivity (hereafter called permittivity).

Knowing the soil permittivity allows modelling the soil influence on the total signal.
In current radiative transfer models targeting snow characteristics, frozen soil permittivity is often set to a constant close to that of dry soil \parencite{Hallikainen1985,Kerr2012}, due to a lack of knowledge on the spatial and temporal variability of frozen soil permittivity.
Soil permittivity greatly depends on soil water content \parencite{Topp1980} where water permittivity is relatively high (\(\varepsilon^\prime \approx 80\) at L-band) compared to dry soil (\(\varepsilon^\prime \approx 2.5\) \parencite{Matzler1998}) or ice (\(\varepsilon^\prime \approx 3.2\) \parencite{Matzler1987}).
Having unfrozen water in the soil greatly increases the permittivity, hence modifying the soil backscattering signal.
Previous work on soil dielectric properties have been made where the dielectric constant is measured as a function of either the temperature and/or the water content and the soil texture \parencite{Cihlar1974,Hoekstra1974,Bobrov2015,Kabir2020}.
Models for soil permittivity (called dielectric mixing models, \parencite{Dobson1985,Hallikainen1985,Mironov2009,Zhang2010}) have also been previously developed.

The goal of this article is thus to present a novel \ac{oecp} with an operating frequency range from \SIrange{0.5}{18}{\giga\hertz} able to measure the permittivity of a flat material or liquid.
The probe aperture (the measurement plane) is large enough to integrate a representative average permittivity over non-homogeneous soil.
The experimental setup required to operate the probe is simple and easily portable, making it perfect for laboratory and field use.
In this study, we present the novel \ac{oecp} and its precision over its full operating range as well as tests to estimate the probe signal penetration depth.
Then, liquids of known permittivity are used to compare the probe's results to theoretical values to validate the probe calibration.
Finally, a protocol was developed to measure the soil permittivity in a controlled varying temperature environment (from \qtyrange{-20}{20}{\degreeCelsius}).
Two different soil types (sand and arctic organic soil) were analyzed following the new protocol.

